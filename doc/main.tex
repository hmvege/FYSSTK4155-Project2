\documentclass[11pt]{article}

\usepackage[utf8]{inputenc}
\usepackage{mathtools}
\usepackage{amsmath}
\usepackage{amsfonts}
\usepackage{enumerate}

% For proper referencing in article
\usepackage{hyperref}
\usepackage{url}

% For figures and graphics'n stuff
\usepackage{graphicx}
\usepackage{caption}
\usepackage{subcaption}
% \usepackage{tabularx}
\usepackage{float}

% For proper appendices
\usepackage[toc,page]{appendix}

% Algorithm packages
\usepackage{algorithm}
\usepackage{algorithmicx}
\usepackage{algpseudocode}

% For bold math symbols
\usepackage{bm}
\usepackage{xcolor}

% For customized hlines in tables
\usepackage{ctable}

% For having latex symbols in section titles
\usepackage{epstopdf}

% For proper citations
% \usepackage[round, authoryear]{natbib}
\usepackage[numbers]{natbib} 

% For fixing large table height
\usepackage{a4wide}

% Remembrance and checking
\newcommand{\husk}[1]{\color{red} #1 \color{black}}
\newcommand{\sjekk}[1]{\color{violet} #1 \color{black}}

\DeclareMathOperator{\sign}{sign}
\DeclareMathOperator*{\argmin}{argmin}
\DeclareMathOperator*{\CO}{\mathcal{C}}


% \title{FYS-STK4155: Project 2}
\title{Exploring the hyperspace of Machine Learning parameters}
\author{Eirik Ramsli Hauge, Joakim Kalsnes, Hans Mathias Mamen Vege}
\date{\today}

\begin{document}
\maketitle

\begin{abstract}
An implementation of a multilayer perceptron neural network and logistic regression is presented, in which the effects of tuning different hyperparameters is made studying the 1D and 2D Ising model. As a verification, comparisons between 1D Ising model applied to ordinary least squares, Ridge and Lasso regression and the MLP is made, to which the MLP appears outperform the linear regression in certain aspects. For the 2D Ising model, we find that the optimal parameters for MLP seems to lie around a constant learning rate $\eta=0.001$, L$^2$ regularization strength $\lambda=0.1$, mini batch size in SGD(Stochastic gradient descent) for $N_\mathrm{mb}=20$, 20 hidden layer neurons. This produces after 500 epochs a accuracy of 
\end{abstract}

% \tableofcontents

%%%%%%%%%%%%%%%%%%%%%%%%%%%%%%%%
\section{Introduction}
With the rapid expansion of computer power, data science is becomming an ever bigger part of modern society. Among the methods used to process data, some of the most popular are regression and classification. Through this paper we will look closely at the use of these methods when applied to data from the Ising model for spins. First of, we will use linear, ridge and lasso regression to estimate the coupling constant for the one-dimensional Ising model. Secondly, we move on to logistic regression and determining the phase of a spin matrix created by the two-dimensional Ising model. The latter problem is a classification problem and we will look further into it by creating a neural net. To learn more about the properties of a neural net, we will start by using the one-dimensional Ising model to find the optimal weights and biases in a regression case. Then we will apply what we learned by repeating the same classification as the logistic case, but this time we will train a neural net with a cross-entropy cost function to perform the task. 

Our work is heavily inspired by the work of Metha et al. \cite{2018arXiv180308823M}, and therefore it is only natural to compare our results to theirs. Lastly, we will give a detailed and in-depth analyzis of the algorithms used and how our home-made algorithms compares to those of Metha et al. and similar calculations done by standard libraries such as scikit-learn or TensorFlow.

%%%%%%%%%%%%%%%%%%%%%%%%%%%%%%%%
% Theory introduction goes here

\section{Theory}
\subsection{The Ising Model}
\subsubsection{1-dimensional Ising model}
\subsubsection{2-dimensional Ising model}
Some text if not it wont compile

\subsection{Logistic regression}
In project 1 we used linear regression to predict a continuous output from a set of inputs (\husk{reference project1}). We used ordinary least squares regression (OLS), Ridge regression and Lasso regression, where the two latter impose a penalty to the OLS. In this project we will reuse the ideas and code of project 1, but we will also use neural networks to predict continuous variables. In addition we study situations where the outcome is discrete rather than continuous. This is a classification problem, and we will use logistic regression to model the probabilities of the classes.\\

\subsection{Logistic regression}
Just like a linear regression model, a logistic regression model computes a weighted sum of the predictor variables, written in matrix notation as $\bm{X}^T\beta$. However, the logistic regression returns the logistic of the this weighted sum as the probabilities. For a classification problem with K classes, the model has the following form (Hastie et al. p.119), \\
\begin{equation}\label{eqT:logreg_def}
\begin{split}
log\frac{Pr(G=1|X=x)}{Pr(G=K|X=x)} &= \beta_{10}\ +\ \beta_{1}^Tx\\
log\frac{Pr(G=2|X=x)}{Pr(G=K|X=x)} &= \beta_{20}\ +\ \beta_{2}^Tx\\
&\vdots\\
log\frac{Pr(G=K-1|X=x)}{Pr(G=K|X=x)} &= \beta_{(K-1)0}\ +\ \beta_{K-1}^Tx\\
\end{split}
\end{equation}
It is arbitrary which class is used in the denominator for the log-odds above. Taking the exponential on both sides and solving for $Pr(G=k|X=x)$ gives the following probabilities:\\
\begin{equation}\label{eqT:logreg_prob}
\begin{split}
Pr(G=k|X=x) &= \frac{exp(\beta_{k0}\ +\ \beta_{k}^Tx)}{1+\sum_{l=1}^{K-1}exp(\beta_{l0}\ +\ \beta_{l}^Tx)},\ k=1,...,K-1,\\
Pr(G=K|X=x) &= \frac{1}{1+\sum_{l=1}^{K-1}exp(\beta_{l0}\ +\ \beta_{l}^Tx)},
\end{split}
\end{equation}
and the probabilities sum to one. The output is then classified as the class with the highest probability.\\

\subsubsection{Fitting logistic regression model}
The usual way of fitting logistic regression models is by maximum likelihood. The log-likelihood for N observations is defined as:\\
\begin{equation}\label{eqT:likelihood_def}
l(\theta)\ =\ \sum_{i=1}^{N}logp_{g_i}(x_i;\theta),\\
\end{equation}
where $p_k(x_i;\theta)\ =\ Pr(G=k|X=x_i;\theta)$ and $\theta\ =\ \{\beta_{10}, \beta_1^T,....., \beta_{(K-1)0}, \beta_{K-1}^T\}$.

One very common classification problem is a situation with binary outcomes, either it happens or it does not. As we see from Equation \ref{eqT:logreg_def} above, setting K=2 simplifies the model considerable, since there will now be only a single linear function. $\theta$ in Equation \ref{eqT:likelihood_def} will also be simplified: $\theta = \beta = \{\beta_{10}, \beta_1^T\}$. The two-class case is what is used in this project, and the following discussion will assume the outcome has two classes.\\
We start by coding the two-class $g_i$ with a 0/1 response $y_i$, where $y_i$ = 1 when $g_i$ = 1, and $y_i$ = 0 when $g_i$ = 2. Next, we let $p_1(x;\theta)\ =\ p(x;\beta)$, and $p_2(x;\theta)\ =\ 1\ -\ p(x;\beta)$. The log-likelihood can then be written
\begin{equation}\label{eqT:loglike_binary}
\begin{split}
l(\beta) &= \sum_{i=1}^N\{y_ilogp(x_i;\beta)+(1-y_i)log(1-p(x_i;\beta))\}\\
 &= \sum_{i=1}^N\{y_ilog\frac{p(x_i;\beta)}{1-p(x_i;\beta)}+log(1-p(x_i;\beta))\}\\
 &= \sum_{i=1}^N\{y_i\beta^Tx_i + log(1-\frac{1}{1+exp(-\beta^Tx_i)}\}\\
 &= \sum_{i=1}^N\{y_i\beta^Tx_i + log(\frac{exp(1}{1+exp(\beta^Tx_i)}\}\\
 &= \sum_{i=1}^N\{y_i\beta^Tx_i - log(1+exp(\beta^Tx_i))\}.
\end{split}
\end{equation}
This is the equation we want to maximize to find the best fit. Following the approach in Géron's book (\husk{reference}), we chose the equivalent approach of minimizing the following\\
\begin{equation}\label{Geron_cost}
J(\beta) = -\frac{1}{N}\sum_{i=1}^N\{y_i\beta^Tx_i - log(1+exp(\beta^Tx_i)).\}
\end{equation}
This is just the negative of Equation \ref{eqT:loglike_binary}, divided by the number of samples. This is our cost function, and dividing by the number of training samples finds the mean cost.

To minimize this cost function we used gradient descent. \sjekk{Gradient descent measures the local gradient of the cost function, with regards to $\beta$ in our case. Since the gradient goes in the direction of fastest increase, we will go in the opposite direction, i.e. negative gradient. We start by choosing random values for $\beta$ (since our cost function is convex any choice should give correct results), calculate the gradient, update the $\beta$ values, and do this iteratively until the algorithm converges to a minimum. The size of the steps is important, and is determined by the learning rate. If the learning rate is too small, we will need many iterations which is time consuming. However, if the learning rate is too high, we might overshoot and miss the minimum. One way to choose the learning rate is too let it depend on the size of the gradient. If the gradient is large, i.e a steep slope, the learning rate can be relatively high. When the gradient is small, the learning rate is also small.}

\sjekk{Returning to the logistic regression problem, the derivative of the cost function is
\begin{equation}\label{eqT:diff_cost}
\begin{split}
\frac{\partial J(\beta)}{\partial \beta} &=-\frac{1}{N}\bm{X}^T(\bm{y}-\bm{p})\\
 &= \frac{1}{N}\bm{X}^T(\bm{p}-\bm{y}),
\end{split}
\end{equation}
where $\bm{X}$ is the $N\times(p+1)$ matrix of $x_i$ values, $\bm{p}$ is the vector of fitted probabilities with $i$th element $p(x_i;\beta)$ and $\bm{y}$ is the vector of $y_i$ values. The new $\beta$ using gradient descent is then\\
$\beta_{new} = \beta_{old}\ -\ \frac{\partial J(\beta)}{\partial \beta}lr$, where $lr$ is the learning rate (step size).} \husk{This is done iteratively until we reach the set max iterations or $\frac{\partial J(\beta)}{\partial \beta}$ is within a given tolerance of zero.}

\sjekk{Like we introduced Lasso and Ridge regression to avoid overfitting in Project 1, we can add a penalty term to the cost function in Equation \ref{Geron_cost}. In our project we used two different penalties: $L1 = \lambda|\beta|$ and $L2 = \lambda||\beta||^2$. When fitting the model we need to include the derivatives of the penalty term in Equation \ref{eqT:diff_cost}.} \husk{The gradient with the penalty term is\\
\begin{equation}
\begin{split}
\frac{\partial J(\beta)}{\partial \beta} &=\frac{1}{N}\bm{X}^T(\bm{p}-\bm{y})\ + \lambda\cdot{sign(\beta)},\ for\ L1\ regularization\\
 &or\\
\frac{\partial J(\beta)}{\partial \beta} &=\frac{1}{N}\bm{X}^T(\bm{p}-\bm{y})\ + \lambda\cdot{2\beta},\ for\ L2\ regularization. 
\end{split}
\end{equation}
}


\subsection{Cost Functions}
\input{theory/cost_functions.tex}

\subsection{Neural Networks}
Some text if not it wont compile


%%%%%%%%%%%%%%%%%%%%%%%%%%%%%%%%
\section{Implementation}
Code can be found on \citep{github-repo}.

%%%%%%%%%%%%%%%%%%%%%%%%%%%%%%%%
\section{Results}
Results for two different cases are being presented, one the one-dimensional Ising Model and another for the the two-dimensional Ising model. We begin with looking at the. 1D Ising model.

\subsection{1D Ising model}
\subsubsection{Fitting with linear regression}
For linear regression we got coefficients of $\bm{J}$ in \eqref{eq:1d-ising-linreg} as the following presented in figure \ref{fig:bias-var-franke},
\begin{figure}[H]
    \centering
    \begin{subfigure}[b]{0.9\textwidth}
        \includegraphics[trim={1.5cm 3.5cm 0 3.5cm},clip, scale=1]{../fig/{regression_ising_1d_heatmap_lambda0.001}.pdf}
        \caption{$\lambda=10^{-3}$}
        \label{fig:linreg-hm-1e-3}
    \end{subfigure} \\
    \begin{subfigure}[b]{0.9\textwidth}
        \includegraphics[trim={1.5cm 3.5cm 0 3.5cm},clip, scale=1]{../fig/{regression_ising_1d_heatmap_lambda0.1}.pdf}
        \caption{$\lambda=10^{-1}$}
        \label{fig:linreg-hm-1e-1}
    \end{subfigure} \\
    \begin{subfigure}[b]{0.9\textwidth}
        \includegraphics[trim={1.5cm 3.5cm 0 3.5cm},clip, scale=1]{../fig/{regression_ising_1d_heatmap_lambda10.0}.pdf}
    \caption{$\lambda=10^{1}$}
        \label{fig:linreg-hm-1e2}
    \end{subfigure}
    \caption{Heat map plots of the $\bm{J}$ in \eqref{eq:1d-ising-linreg} retrieved from OLS, Ridge and Lasso. Gathered using $N_\mathrm{train}=5000$.}
    \label{fig:bias-var-franke}
\end{figure}

The $R^2$ score of the OLS, Ridge and Lasso can be seen in figure \ref{fig:linreg-r2},
\begin{figure}[H]
    \centering
    \includegraphics[scale=1.0]{../fig/r2_ols_ridge_lasso.pdf}
    \caption{$R^2$ score for different Ordinary Least Squares(OLS), Ridge and Lasso regression. Retrieved $N_\mathrm{train}=5000$ on a 1D Ising model of size $L=20$.}
    \label{fig:linreg-r2}
\end{figure}

The bias-variance decomposition for Ridge and Lasso using bootstrap and cross validation can be viewed in figure \ref{fig:linreg-bias-variance-decomp-ridge} and \ref{fig:linreg-bias-variance-decomp-lasso}.

\begin{figure}[H]
    \centering
    \begin{subfigure}[b]{0.5\textwidth}
        \centering
        \includegraphics[scale=0.5]{../fig/ridge_bs_bias_variance_analysis.pdf}
        \caption{Bootstrap.}
        \label{fig:linreg-bias-variance-decomp-bs-ridge}
    \end{subfigure}%
    \begin{subfigure}[b]{0.5\textwidth}
        \centering
        \includegraphics[scale=0.5]{../fig/ridge_cv_bias_variance_analysis.pdf}
        \caption{$k$-fold Cross Validation.}
        \label{fig:linreg-bias-variance-decomp-cv-ridge}
    \end{subfigure}
    \caption{A bias-variance decomposition of Ridge regression using bootstrapping\ref{fig:linreg-bias-variance-decomp-bs-ridge} and cross-validation\ref{fig:linreg-bias-variance-decomp-cv-ridge}.}
    \label{fig:linreg-bias-variance-decomp-ridge}
\end{figure}

\begin{figure}[H]
    \centering
    \begin{subfigure}[b]{0.5\textwidth}
        \centering
        \includegraphics[scale=0.5]{../fig/lasso_bs_bias_variance_analysis.pdf}
        \caption{Bootstrap.}
        \label{fig:linreg-bias-variance-decomp-bs-lasso}
    \end{subfigure}%
    \begin{subfigure}[b]{0.5\textwidth}
        \centering
        \includegraphics[scale=0.5]{../fig/lasso_cv_bias_variance_analysis.pdf}
        \caption{$k$-fold Cross Validation.}
        \label{fig:linreg-bias-variance-decomp-cv-lasso}
    \end{subfigure}
    \caption{A bias-variance decomposition of Lasso regression using bootstrapping\ref{fig:linreg-bias-variance-decomp-bs-lasso} and cross-validation\ref{fig:linreg-bias-variance-decomp-cv-lasso}.}
    \label{fig:linreg-bias-variance-decomp-lasso}
\end{figure}

\subsubsection{Fitting with a neural network}
By setting the output activation function to the identity and by having zero hidden layers, we are essentially performing a regression analysis on the 1D Ising model. We generate the same amount of data by inputing the same RNG(random number generator) seed. A fit using $N_\mathrm{train}=400$ and $N_\mathrm{train}=5000$ for $\lambda=10^{-3}, 10^{-1}, 10^1$ can be seen in figure \ref{fig:mlp_coefs}.

\begin{figure}[H]
    \centering
    \begin{subfigure}[b]{0.4\textwidth}
        \centering
        \includegraphics[trim={1.5cm 3.5cm 0 3.5cm},clip, scale=0.6]{../fig/{mlp_ising_1d_heatmap_lambda0.0001_N800}.pdf}
        \caption{$N_\mathrm{train}=400$, $\lambda=10^{-3}$}
        \label{fig:mlp-reg-heatmap400-lmb-3}
    \end{subfigure} \qquad \qquad \qquad
    \begin{subfigure}[b]{0.4\textwidth}
        \centering
        \includegraphics[trim={1.5cm 3.5cm 0 3.5cm},clip, scale=0.6]{../fig/{mlp_ising_1d_heatmap_lambda0.0001_N100000}.pdf}
        \caption{$N_\mathrm{train}=5000$, $\lambda=10^{-3}$}
        \label{fig:mlp-reg-heatmap5000-lmb-3}
    \end{subfigure} \\
        \begin{subfigure}[b]{0.4\textwidth}
        \centering
        \includegraphics[trim={1.5cm 3.5cm 0 3.5cm},clip, scale=0.6]{../fig/{mlp_ising_1d_heatmap_lambda0.1_N800}.pdf}
        \caption{$N_\mathrm{train}=400$, $\lambda=10^{-1}$}
        \label{fig:mlp-reg-heatmap400-lmb-1}
    \end{subfigure} \qquad \qquad \qquad
    \begin{subfigure}[b]{0.4\textwidth}
        \centering
        \includegraphics[trim={1.5cm 3.5cm 0 3.5cm},clip, scale=0.6]{../fig/{mlp_ising_1d_heatmap_lambda0.1_N100000}.pdf}
        \caption{$N_\mathrm{train}=5000$, $\lambda=10^{-1}$}
        \label{fig:mlp-reg-heatmap5000-lmb-1}
    \end{subfigure} \\
        \begin{subfigure}[b]{0.4\textwidth}
        \centering
        \includegraphics[trim={1.5cm 3.5cm 0 3.5cm},clip, scale=0.6]{../fig/{mlp_ising_1d_heatmap_lambda10.0_N800}.pdf}
        \caption{$N_\mathrm{train}=400$, $\lambda=10^1$}
        \label{fig:mlp-reg-heatmap400-lmb1}
    \end{subfigure} \qquad \qquad \qquad
    \begin{subfigure}[b]{0.4\textwidth}
        \centering
        \includegraphics[trim={1.5cm 3.5cm 0 3.5cm},clip, scale=0.6]{../fig/{mlp_ising_1d_heatmap_lambda10.0_N100000}.pdf}
        \caption{$N_\mathrm{train}=5000$, $\lambda=10^1$}
        \label{fig:mlp-reg-heatmap5000-lmb1}
    \end{subfigure} \\
    \caption{Heat map plot of the coefficients of $\bm{J}$ in \eqref{eq:1d-ising-linreg} using neural networks with different regularizations for $\lambda=10^{-3}, 10^{-1}, 10^1$.}
    \label{fig:mlp-coefs}
\end{figure}

The $R^2$ score of the neural network using L$^1$, L$^2$ and no regularization can be seen in figure \ref{fig:mlp-r2},
\begin{figure}[H]
    \centering
    \begin{subfigure}[b]{0.5\textwidth}
        \centering
        \includegraphics[scale=0.5]{../fig/{mlp_r2_ols_ridge_lasso800}.pdf}
        \caption{$N_\mathrm{train}=400$}
        \label{fig:mlp-r2-800}
    \end{subfigure}%
    \begin{subfigure}[b]{0.5\textwidth}
        \centering
        \includegraphics[scale=0.5]{../fig/{mlp_r2_ols_ridge_lasso100000}.pdf}
        \caption{$N_\mathrm{train}=5000$}
        \label{fig:mlp-r2-5000}
    \end{subfigure}
    \caption{$R^2$ score for the neural network using L$^1$ (Lasso), L$^2$ (Ridge) and no regularization (OLS). Retrieved $N_\mathrm{train}=400$ on the left and $N_\mathrm{train}=5000$ on the right, for a 1D Ising model of size $L=20$.}
    \label{fig:mlp-r2}
\end{figure}

% TODO: rerun mlp regression with N_samples = 10000 as I run for too much :|

\subsection{2D Ising model}
As stated in the section about the 2D Ising model \ref{sec:2d-ising-model}, the classification will focus on evaluating the phases of different lattice configurations, and wetter or not it is below or above a critical temperature. We begin by listing the results from the logistic regression.
\subsubsection{Classification through logistic regression}
In logistic regression we investigated the behavior of the classification and compared it to that of SciKit Learn\cite{scikit-learn}, using the standard logistic regression method\footnote{See \href{https://scikit-learn.org/stable/modules/generated/sklearn.linear_model.LogisticRegression.html}{Logistic Regression documentation}} and 
SciKit Learn's SGD(Stochastic Gradient Descent) implementation \footnote{See \href{https://scikit-learn.org/stable/modules/generated/sklearn.linear_model.SGDClassifier.html}{SGD documentation}}. This gave the results found in figure \ref{fig:logreg-accuracy-sklearn-comparison}.

\begin{figure}[H]
    \centering
    \includegraphics[scale=1.0]{../fig/logistic_accuracy_sklearn_comparison.pdf}
    \caption{The accuracy for our implementation of logistic regression versus that of SciKit learn.}
    \label{fig:logreg-accuracy-sklearn-comparison}
\end{figure}

\subsubsection{Classification through neural networks}
For classifying the states through a neural network, we looked at several different hyper parameters. All runs were made using $N_{samples}=10000$ except stated other wise. The training percent was 0.5. We start by comparing two different cost functions and their layer outputs,
\begin{itemize}
    \item Cross entropy with softmax layer output\eqref{eq:ce-mlp-cost}
    \item MSE with sigmoidal layer output.\eqref{eq:mse-mlp-cost}
\end{itemize}
These cost functions following behavior for epochs seen in following figure \ref{fig:mlp-cost-function-comparison},
\begin{figure}[H]
    \centering
    \includegraphics[scale=1.0]{../fig/mlp_epoch_cost_functions.pdf}
    \caption{A comparison in accuracy scores between the MSE and (CE) Cross Entropy loss functions over 500 epochs. The output layer for MSE is sigmoidal, the output layer for CE is softmax. The learning parameter was $\eta=0.001$ and we used the inverse learning rate\eqref{eq:inverse-eta}.}
    \label{fig:mlp-cost-function-comparison}
\end{figure}

We then wish to to investigate the effects of having different initial weights. Given the initial weights \textit{large} and \textit{default} as listed in section \ref{sec:nn-weights}, we get the results as seen in figure \ref{fig:mlp-epoch-init-weights},
\begin{figure}[H]
    \centering
    \includegraphics[scale=1.0]{../fig/mlp_epoch_weight_inits.pdf}
    \caption{A comparison in accuracy scores between the initial weights \textit{large} and \textit{default} as listed in section \ref{sec:nn-weights}. The run was for 500 epochs. The cost function was set to cross entropy and had softmax output activation. The learning parameter was $\eta=0.001$ and we used the inverse learning rate\eqref{eq:inverse-eta}.}
    \label{fig:mlp-epoch-init-weights}
\end{figure}

skriv inn lambda ting her !=")#(097 21841 \\\\}Ʒ׶"

An investigation into different layer activations\ref{sec:layer-acts} was performed for both the MSE- and the CE-cost function. The results from MSE can be seen in figure 
\begin{figure}[H]
    \centering
    \includegraphics[scale=1.0]{../fig/mlp_epoch_activations_mse.pdf}
    \caption{A comparison in accuracy scores between the hidden layer activation functions(see section \ref{sec:layer-acts}) for MSE as cost function. The run was for 500 epochs. The learning rate was set with the inverse learning rate \eqref{eq:inverse-eta} with an $\eta_0=0.001$ and $\lambda=0.0$.}
    \label{fig:mlp-epoch-activations-mse}
\end{figure}
\begin{figure}[H]
    \centering
    \includegraphics[scale=1.0]{../fig/mlp_epoch_activations_log_loss.pdf}
    \caption{A comparison in accuracy scores between the hidden layer activation functions(see section \ref{sec:layer-acts}) for cross entropy as cost function. The run was for 500 epochs. The learning rate was set with the inverse learning rate \eqref{eq:inverse-eta} with an $\eta_0=0.001$ and $\lambda=0.0$.}
    \label{fig:mlp-epoch-activations-log-loss}
\end{figure}

We then move on to an investigation for different L$^2$ regularization strengths $\lambda$ versus different constant learning rates $\eta$. A run with 500 epochs, cross entropy and sigmoidal hidden layer activation can be seen in figure \ref{fig:mlp-eta-lambda},
\begin{figure}[H]
    \centering
    \includegraphics[scale=1.0]{../fig/mlp_lambda_eta.pdf}
    \caption{The accuracy as function of the L$^2$ regularization parameter $\lambda$ and constant training rate $\eta$. The run was for 500 epochs and with cross entropy as cost function, softmax output and sigmoidal hidden layer activation. The hidden layer was 10 neurons large.}
    \label{fig:mlp-eta-lambda}
\end{figure}

A comparison of the accuracy score\eqref{eq:mlp-accuracy} as a function of L$^2$ regularization parameter and hidden layer size(the neurons) can be viewed in figure \ref{fig:mlp-lambda-neurons}.
\begin{figure}[H]
    \centering
    \includegraphics[scale=1.0]{../fig/mlp_lambda_neurons.pdf}
    \caption{The accuracy score as function of the L$^2$ regularization parameter $\lambda$ and the number of neurons. The run was for 500 epochs and with cross entropy as cost function, softmax output and sigmoidal hidden layer activation. The learning rate was set with the inverse learning rate \eqref{eq:inverse-eta} with an $\eta_0=0.001$.}
    \label{fig:mlp-lambda-neurons}
\end{figure}

The accuracy score\eqref{eq:mlp-accuracy} as a function of the hidden layer size(the neurons) and the training data size as percentage of of a $N_\mathrm{samples}=10000$ training data, can be viewed in figure \ref{fig:mlp-neurons-ts}.
\begin{figure}[H]
    \centering
    \includegraphics[scale=1.0]{../fig/mlp_neurons_training_size.pdf}
    \caption{The accuracy score as function of the number of neurons and the training data size percentage of $N_\mathrm{samples}=10000$. The run was for 500 epochs and with cross entropy as cost function, softmax output and sigmoidal hidden layer activation. The learning rate was set with the inverse learning rate \eqref{eq:inverse-eta} with an $\eta_0=0.001$ and $\lambda=0.0$. }
    \label{fig:mlp-neurons-ts}
\end{figure}

The accuracy score\eqref{eq:mlp-accuracy} as a function of the hidden layer size(the neurons) and the learning rate $\eta$, can be viewed in figure \ref{fig:mlp-neurons-ts}.
\begin{figure}[H]
    \centering
    \includegraphics[scale=1.0]{../fig/mlp_neurons_eta.pdf}
    \caption{The accuracy score as function of the number of neurons and the learning rate $\eta$. The run was for 500 epochs and with cross entropy as cost function, softmax output and sigmoidal hidden layer activation. The regularization strength was set to $\lambda=0.0$}.
    \label{fig:mlp-neurons-eta}
\end{figure}

The accuracy score\eqref{eq:mlp-accuracy} as a function of L$^2$ regularization strength $\lambda$ and the mini batch size in the SGD\ref{alg:sgd}, can be viewed in figure \ref{fig:mlp-lambda-mb}.
\begin{figure}[H]
    \centering
    \includegraphics[scale=1.0]{../fig/mlp_lambda_mini_batch_size.pdf}
    \caption{The accuracy score as function of the  L$^2$ regularization strength $\lambda$ and the mini batch size in the SGD\ref{alg:sgd}. The run was for 500 epochs and with cross entropy as cost function, softmax output and sigmoidal hidden layer activation and the inverse learning rate \eqref{eq:inverse-eta} with $\eta_0=0.001$.}
    \label{fig:mlp-lambda-mb}
\end{figure}


%%%%%%%%%%%%%%%%%%%%%%%%%%%%%%%%
% \documentclass[11pt]{article}

% \usepackage[utf8]{inputenc}
% \usepackage{mathtools}
% \usepackage{amsmath}
% \usepackage{amsfonts}
% \usepackage{enumerate}

% % For proper referencing in article
% \usepackage{hyperref}
% \usepackage{url}

% % For figures and graphics'n stuff
% \usepackage{graphicx}
% \usepackage{caption}
% \usepackage{subcaption}
% % \usepackage{tabularx}
% \usepackage{float}

% % For proper appendices
% \usepackage[toc,page]{appendix}

% % Algorithm packages
% \usepackage{algorithm}
% \usepackage{algorithmicx}
% \usepackage{algpseudocode}

% % For bold math symbols
% \usepackage{bm}
% \usepackage{xcolor}

% % For customized hlines in tables
% \usepackage{ctable}

% % For having latex symbols in section titles
% \usepackage{epstopdf}

% % For proper citations
% % \usepackage[round, authoryear]{natbib}
% \usepackage[numbers]{natbib} 

% % For fixing large table height
% \usepackage{a4wide}

% % Remembrance and checking
% \newcommand{\husk}[1]{\color{red} #1 \color{black}}
% \newcommand{\sjekk}[1]{\color{violet} #1 \color{black}}

% \DeclareMathOperator{\sign}{sign}
% \DeclareMathOperator*{\argmin}{argmin}
% \DeclareMathOperator*{\CO}{\mathcal{C}}


% % \title{FYS-STK4155: Project 2}
% \title{Exploring the hyperspace of Machine Learning parameters}
% \author{Eirik Ramsli Hauge, Joakim Kalsnes, Hans Mathias Mamen Vege}
% \date{\today}

% \begin{document}
\section{Discussion}

\subsection{1D Ising model}
\subsubsection{Fitting with linear regression}
By examining the results from figure \ref{fig:bias-var-franke} illustrates the effect of different $\lambda$-values. The diagonal line illustrates the particle we are looking at and the pixels to its immediate left and right are the nearest neighbours. It is clear from this figure that even though we assumed that all particles interact with each other, only the neighbouring particles will have a profound effect on each other. For Lasso, it seems that the optimal $\lambda$ is either $10^{-3}$ or $10^{-1}$. This is agreeable with the results of Metha et al \cite{2018arXiv180308823M} who found $10^{-2}$ to be the optimal parameter. For Ridge and OLS we can not seem to see any big difference between the different $\lambda$'s. The inability to find the best $\lambda$ for OLS and Ridge and the fact that we do not have a clear optimal parameter for Lasso indicates that we should have gathered more data. \\ \\
However, by looking at figure \ref{fig:linreg-r2} we can see that there is little change in the R2-score for different $\lambda$-values, indicating that it would not matter if we generated many more plots. They would most likely stay the same. It is also clear that Lasso is the most sensitive of the models, loosing ground to the other regression methods already at $\lambda = 10^{-1}$. The R2-score for Ridge starts to decline at about $\lambda = 10^{3}$, while OLS seems to maintain its quality over all $\lambda$'s. \\ \\
This trend is further confirmed by figure \ref{fig:fig:linreg-bias-variance-decomp-ridge} and \ref{fig:linreg-bias-variance-decomp-lasso} where we have used Bootstrap and $k$-fold validation to verify our results. As we can see, the variance stays about the same for all $\lambda$-values, but MSE and therefore, the bias increases. This shift in bias also happens around $\lambda = 10^{-1}$ for Lasso and $\lambda = 10^{3}$ for Ridge as the R2-score did as well.
\subsubsection{Fitting with a neural network}
Repeating the calculations using a neural net with zero hidden layers and the identity function as the activation function gives us the opportunity to repeat the above calculations using a neural net instead. As we can see from figure \ref{fig:mlp-coefs}, the difference in $N_{\text{train}}$ had little effect, but the change in $\lambda$ was significant. The change between OLS, Ridge and Lasso also represent a change in regularisation from no regularisation to L$^2$ to L$^1$ respectively. \\
In contrast to the linear regression case, there is a difference between $\lambda = 10^{-3}$ and $\lambda = 10^{-1}$. The diagonal lines once removed which signifies the nearest neighbours are about as strong for both parameters, however, there is a noticeable decrease in noise for the $\lambda = 10^{-1}$ case. Furthermore, we must note that the use of the parameter $\lambda = 10^{1}$ is more devastating for Ridge than Lasso this time around. Lasso looses a lot of information, but for Ridge the significant of nearest neighbours disappears completely. The same trend is once again evident for the R2-score plotted in figure \ref{fig:mlp-r2}. However, with this representation we can see that increasing $N_{\text{train}}$ decreases the difference between the test and train data. This is as expected.
\subsection{2D Ising model}
\subsubsection{Classification through logistic regression}
\subsubsection{Classification through neural networks}
% \end{document}

%%%%%%%%%%%%%%%%%%%%%%%%%%%%%%%%
\documentclass[11pt]{article}

\usepackage[utf8]{inputenc}
\usepackage{mathtools}
\usepackage{amsmath}
\usepackage{amsfonts}
\usepackage{enumerate}

% For proper referencing in article
\usepackage{hyperref}
\usepackage{url}

% For figures and graphics'n stuff
\usepackage{graphicx}
\usepackage{caption}
\usepackage{subcaption}
% \usepackage{tabularx}
\usepackage{float}

% For proper appendices
\usepackage[toc,page]{appendix}

% Algorithm packages
\usepackage{algorithm}
\usepackage{algorithmicx}
\usepackage{algpseudocode}

% For bold math symbols
\usepackage{bm}
\usepackage{xcolor}

% For customized hlines in tables
\usepackage{ctable}

% For having latex symbols in section titles
\usepackage{epstopdf}

% For proper citations
% \usepackage[round, authoryear]{natbib}
\usepackage[numbers]{natbib} 

% For fixing large table height
\usepackage{a4wide}

% Remembrance and checking
\newcommand{\husk}[1]{\color{red} #1 \color{black}}
\newcommand{\sjekk}[1]{\color{violet} #1 \color{black}}

\DeclareMathOperator{\sign}{sign}
\DeclareMathOperator*{\argmin}{argmin}
\DeclareMathOperator*{\CO}{\mathcal{C}}


% \title{FYS-STK4155: Project 2}
\title{Exploring the hyperspace of Machine Learning parameters}
\author{Eirik Ramsli Hauge, Joakim Kalsnes, Hans Mathias Mamen Vege}
\date{\today}

\begin{document}
\section{Conclusion}
We found that although both linear and logistic regression can be used, neural nets that are correctly tuned will give better scores. From our experiment, this is evident from the increase in R2 and accuracy score when comparing linear and logistic regression to their neural net counterparts respectively. For our linear regression, we found similar results to previous experiments and our results for logistic regression were different from those of Metha et al.. The reason for this difference is left to future experiments to decipher, but as it stands now, the implemented logistic regression was the superior method. For our neural net, the optimal parameters were a learning rate of $\eta = 1.0e-02$, $\lambda = 1.0e-02$, 20 neurons, 500 epochs and a mini batch size of 30. It was also evident that a neural net with $\tanh$ as activation function was the optimal neural net in our case.
\end{document}

%%%%%%%%%%%%%%%%%%%%%%%%%%%%%%%%
% \begin{appendices}
Put stuff like Bootstrapping, kfold CV, OLS/Ridge/Lasso regression here.

\section{A refresher on linear regression}
% Move this to appendix!
\subsection{Ridge regression}
\subsection{Lasso regression}

\section{Bootstrapping}
\section{\texorpdfstring{$k$}{k}-fold Cross-Validation}
\end{appendices}

%%%%%%%%%%%%%%%%%%%%%%%%%%%%%%%%
\bibliographystyle{plainnat}
\bibliography{bibliography/lib.bib}


\end{document}
