\section{Introduction}
With the rapid expansion of computer power, data science is becomming an ever bigger part of modern society. Among the methods used to process data, some of the most popular are regression and classification. Through this paper we will look closely at the use of these methods when applied to data from the Ising model for spins. First of, we will use linear, ridge and lasso regression to estimate the coupling constant for the one-dimensional Ising model. Secondly, we move on to logistic regression and determining the phase of a spin matrix created by the two-dimensional Ising model. The latter problem is a classification problem and we will look further into it by creating a neural net. To learn more about the properties of a neural net, we will start by using the one-dimensional Ising model to find the optimal weights and biases in a regression case. Then we will apply what we learned by repeating the same classification as the logistic case, but this time we will train a neural net with a cross-entropy cost function to perform the task. 

Our work is heavily inspired by the work of Metha et al. \cite{2018arXiv180308823M}, and therefore it is only natural to compare our results to theirs. Lastly, we will give a detailed and in-depth analyzis of the algorithms used and how our home-made algorithms compares to those of Metha et al. and similar calculations done by standard libraries such as scikit-learn or TensorFlow.